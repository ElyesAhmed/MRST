\documentclass[11pt]{amsart}
\usepackage[abbreviate=false,sorting=none,doi=false]{biblatex}
\addbibresource{refs.bib}

\usepackage{xspace}
\usepackage{amsmath}
\usepackage{amsfonts}
\usepackage{latexsym}
\usepackage{graphicx}
\usepackage{bm}

% 
% =======================================================================
% Formatting tools
% =======================================================================
% Gentle spacing after macros
% Reference: The LaTeX Companion, p.50
\newcommand{\Real}{\mathbb R}
\newcommand{\eg}{e.g.,\xspace}
\newcommand{\ie}{i.e.,\xspace}
\newcommand{\etc}{etc.\@\xspace}
\newcommand{\etal}{et~al.\@\xspace}
% 
% 
% =======================================================================
% Common Mathematics
% =======================================================================
\newcommand{\vect}[1]{\boldsymbol{#1}}
\newcommand{\mat}[1]{\boldsymbol{#1}}
\newcommand{\transp}[1]{{#1}^{\ensuremath{\mathsf{T}}}}
% 
% =======================================================================
% Math Operators
% =======================================================================
\DeclareMathOperator{\erfc}{erfc}
\DeclareMathOperator{\spanop}{span}
\DeclareMathOperator{\rank}{rank}
\DeclareMathOperator{\tr}{tr}
\DeclareMathOperator{\diag}{diag}
\DeclareMathOperator{\Grad}{Grad}
\DeclareMathOperator{\Dive}{Div}
% 
\DeclareMathOperator{\GRAD}{\scshape Grad}
\DeclareMathOperator{\DIVE}{\scshape Div}
% 
\newcommand{\grad}{\nabla}
\newcommand{\dive}{\nabla\cdot}
\newcommand{\curl}{\nabla\times}


\newcommand{\cads}{c^a}
\newcommand{\chatads}{\hat{c}^a}
\newcommand{\eff}{\mathrm{eff}}
\newcommand{\cmax}{c_{\max}}
\newcommand{\Sb}{{\bm{S}}}
\newcommand{\cb}{{\bm{c}}}
\newcommand{\Kb}{{\bm{K}}}
\newcommand{\kb}{{\bm{k}}}
\newcommand{\nb}{{\bm{n}}}
\newcommand{\ub}{{\bm{u}}}
\newcommand{\ubmax}{\ub_{\max}}
\newcommand{\abs}[1]{\left| #1\right|}
\newcommand{\norm}[1]{\left\| #1\right\|}

\newcommand{\fracpar}[2]{\frac{\partial #1}{\partial #2}}

\setlength{\parindent}{0cm}
\setlength{\parskip}{5mm}

\begin{document}

\title[Polymer model]{The polymer model in MRST}

\maketitle

\begin{abstract}
  In this document, we describe briefly the polymer model which is implemented in MRST. 
\end{abstract}

The \texttt{ad-core} module contains a solver for a black-oil system with polymer which includes
shear-thinning effect. Here, we limit our presentation to the two-phase case and do not cover the
shear-thinning effect.

Our starting point is the mass conservation equations for oil, water, and polymer
\begin{align}
  \label{eq:conservation_laws1}
  \frac{\partial}{\partial t} (\rho_{\alpha}\phi S_\alpha) +
  \nabla\cdot(\rho_{\alpha}\vec{v}_\alpha)
  & = 0, \qquad \alpha\in\{w,o\}, \\
  \label{eq:conservation_laws2}
  \frac{\partial}{\partial t} (\rho_{w}\phi S_w c)+\nabla\cdot(c \rho_{w} \vec{v}_{wp}) & = 0.
\end{align}
Here, $\rho_{\alpha}$, $\vec{v}_\alpha$, and $S_\alpha$ denote density, velocity, and saturation of
the phase $\alpha$. The porosity is denoted by $\phi$ and is assumed to be a function $\phi(p)$ of
pressure only, $c$ is the polymer concentration, and $\vec{v}_{wp}$ the velocity of water containing
diluted polymer. Sources and sinks may be included in a manner equivalent to boundary conditions,
and are left out of the above equations.

To model the viscosity change of the mixture, we use the Todd--Longstaff model \cite{TL72:jpt}.
This model introduces a mixing parameter $\omega\in[0, 1]$ that takes into account the degree of
mixing of polymer into water. Assuming that the viscosity $\mu_m$ of a fully mixed polymer solution
is a function of the concentration, the effective polymer viscosity is defined as
\begin{equation}
  \label{eq:defmupeff}
  \mu_{p,\eff}=\mu_m(c)^\omega\mu_p^{1-\omega}\quad\text{ with }\quad  \mu_p=\mu_m(c_{\max}).
\end{equation}
The viscosity of the partially mixed water is given in a similar way by
\begin{equation}
  \label{eq:defmuwe}
  \mu_{w,e}=\mu_m(c)^\omega\mu_w^{1-\omega}.
\end{equation}
The effective water viscosity $\mu_{w,\eff}$ is defined by interpolating linearly between
the inverse of the effective polymer viscosity and the partially mixed water viscosity
\begin{equation}
  \label{eq:defmuweff}
  \frac{1}{\mu_{w,\eff}}=\frac{1-c/\cmax}{\mu_{w,e}}+\frac{c/\cmax}{\mu_{p,\eff}}.
\end{equation}
To take the incomplete mixing into account, we introduce the velocity of water that contains
polymer, which we denote $\vec{v}_{wp}$. For this part of the water phase, the relative permeability
is assumed to be equal to $k_{rw}$ and the viscosity is equal to $\mu_{p,\eff}$.  We also consider
the total water velocity, which we still denote $\vec{v}_{w}$ and for which the viscosity is given
by $\mu_{w,\eff}$. Darcy's law then gives us
\begin{align}
  \label{eq:uwdarc}
  \vec{v}_{w} &=-\frac{k_{rw}}{\mu_{w,\eff}R_k(\cads)}\Kb(\nabla p-\rho_w g\nabla z),\\
  \label{eq:uwpdarc}
  \vec{v}_{wp} &=-\frac{k_{rw}}{\mu_{p,\eff}R_k(\cads)}\Kb(\nabla p-\rho_w g\nabla z) =
  m(c)\vec{v}_w,
\end{align}
as we assume that the presence of polymer does not affect the pressure and the density. The polymer
mobility factor $m(c)$ is defined as
\begin{equation*}
  m(c) = \frac{\mu_{w,\eff}}{\mu_{p,\eff}} 
\end{equation*}
and, after some simplifications, we get
\begin{equation}
  \label{eq:def_m}
  m(c) = \Bigl[\Bigl(1 -\frac{c}{\cmax}\Bigr)
  \Bigl(\frac{\mu_p}{\mu_w}\Bigr)^{1-\omega} + \frac{c}{\cmax}\Bigr]^{-1}.  
\end{equation}
The function $R_k(\cads)$ denotes the actual resistance factor and is a non-decreasing function
which models the reduction of the permeability of the rock to the water phase due to the presence of
absorbed polymer. The concentration of absorbed polymer is denoted by $\cads$. We introduce the
total flux as $\vec{v} = \vec{v}_w + \vec{v}_o$. We have
\begin{equation*}
  \vec{v} = -(\lambda_w + \lambda_o)\Kb\nabla p + g(\lambda_w\rho_w + \lambda_o\rho_o)\Kb\nabla z 
\end{equation*}
and, after some computation, we obtain the following expression of $\vec{v}_\alpha$ as a function of
$\vec{v}$
\begin{equation}
  \label{eq:valphav}
  \vec{v}_w = f_w \vec{v} + \vec{v_g}\quad\text{ and }\quad  \vec{v}_o = f_o \vec{v} - \vec{v_g}
\end{equation}
with
\begin{equation}
  \label{eq:gravterm}
  \vec{v_g} = \frac{\lambda_w\lambda_{o}}{\lambda_w+\lambda_{o}}(\rho_w-\rho_{o})g\Kb\nabla z.
\end{equation}
Here, $\lambda_\alpha$ denotes the mobility of phase $\alpha$, i.e.,
\begin{equation*}
  \lambda_w = \frac{k_{rw}}{\mu_{w,\eff}R_k(\cads)}\quad\text{ and }\quad  \lambda_w = \frac{k_{ro}}{\mu_{o}},
\end{equation*}
and $f_\alpha$ corresponds to the fractional flow, $f_\alpha =\lambda_{\alpha}/(\lambda_{w} +
\lambda_{o})$. The time scale of adsorption is much larger than that of mass transport and we will
assume that adsorption takes place instantaneously so that $\cads$ is a function of $c$ only.  The
reference rock density is $\rho_{r, \text{ref}}$ and the reference porosity $\phi_\text{ref}$.  The
adsorption of polymer is then taken into account by replacing \eqref{eq:conservation_laws2} by
\begin{equation}
  \label{eq:conservation_laws2_withads}
  \frac{\partial}{\partial t} (\rho_{w}\phi S_w c)+
  \fracpar{}{t}\left(\rho_{r, \text{ref}}(1-\phi_\text{ref})\cads\right)+\nabla\cdot(c
  \rho_{w} \vec{v}_{wp})  = 0.
\end{equation}
It is natural to assume that $\cads$ is an increasing function of $c$. Polymer
cannot reach the smallest pores and, as a result, the effective pore volume for
the polymer solution is smaller than the pore volume of the rock. The polymer
mobility factor becomes
\begin{equation}
  \label{eq:defmdps}
  m(c) = \frac{\mu_{w,\eff}}{\mu_{p,\eff}} = \Bigl[\Bigl(1 -\frac{c}{\cmax}\Bigr)
  \Bigl(\frac{\mu_p}{\mu_w}\Bigr)^{1-\omega} + \frac{c}{\cmax}\Bigr]^{-1}.
\end{equation}
Finally the modeling equations are
\begin{subequations}
  \label{eq:goveq}
  \begin{equation}
    \label{eq:goveq1}
    \frac{\partial}{\partial t} (\rho_{\alpha}\phi S_\alpha) +
    \nabla\cdot(\rho_{\alpha}\vec{v}_\alpha)
    = 0,
  \end{equation}
  for $\alpha\in\{w,o\}$,
  \begin{equation}
    \label{eq:goveq2}
    \frac{\partial}{\partial t} (\rho_{w}\phi S_w c)+
    \fracpar{}{t}\left(\rho_{r, \text{ref}}(1-\phi_\text{ref})\cads\right)+\nabla\cdot(c
    \rho_{w} \vec{v}_{wp}) = 0.
  \end{equation}
\end{subequations}
where $\vec{v}_\alpha$ and $\vec{v}_{wp}$ are defined in \eqref{eq:uwdarc} and \eqref{eq:uwpdarc}
using \eqref{eq:defmupeff}, \eqref{eq:defmuwe} and \eqref{eq:newdefmuweff}.

\printbibliography

\end{document}
