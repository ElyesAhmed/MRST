\documentclass[11pt]{amsart}
\usepackage[abbreviate=false,sorting=none,doi=false]{biblatex}
\addbibresource{refs.bib}

\usepackage{xspace}
\usepackage{amsmath}
\usepackage{amsfonts}
\usepackage{latexsym}
\usepackage{graphicx}
\usepackage{bm}
\usepackage{siunitx}

% 
% =======================================================================
% Formatting tools
% =======================================================================
% Gentle spacing after macros
% Reference: The LaTeX Companion, p.50
\newcommand{\Real}{\mathbb R}
\newcommand{\eg}{e.g.,\xspace}
\newcommand{\ie}{i.e.,\xspace}
\newcommand{\etc}{etc.\@\xspace}
\newcommand{\etal}{et~al.\@\xspace}
% 
% 
% =======================================================================
% Common Mathematics
% =======================================================================
\newcommand{\vect}[1]{\boldsymbol{#1}}
\newcommand{\mat}[1]{\boldsymbol{#1}}
\newcommand{\transp}[1]{{#1}^{\ensuremath{\mathsf{T}}}}
% 
% =======================================================================
% Math Operators
% =======================================================================
\DeclareMathOperator{\erfc}{erfc}
\DeclareMathOperator{\spanop}{span}
\DeclareMathOperator{\rank}{rank}
\DeclareMathOperator{\tr}{tr}
\DeclareMathOperator{\diag}{diag}
\DeclareMathOperator{\Grad}{Grad}
\DeclareMathOperator{\Dive}{Div}
% 
\DeclareMathOperator{\GRAD}{\scshape Grad}
\DeclareMathOperator{\DIVE}{\scshape Div}
% 
\newcommand{\grad}{\nabla}
\newcommand{\dive}{\nabla\cdot}
\newcommand{\curl}{\nabla\times}


\newcommand{\cads}{c^a}
\newcommand{\chatads}{\hat{c}^a}
\newcommand{\eff}{\mathrm{eff}}
\newcommand{\cmax}{c_{\max}}
\newcommand{\Sb}{{\bm{S}}}
\newcommand{\cb}{{\bm{c}}}
\newcommand{\Kb}{{\bm{K}}}
\newcommand{\kb}{{\bm{k}}}
\newcommand{\nb}{{\bm{n}}}
\newcommand{\ub}{{\bm{u}}}
\newcommand{\ubmax}{\ub_{\max}}
\newcommand{\abs}[1]{\left| #1\right|}
\newcommand{\norm}[1]{\left\| #1\right\|}

\newcommand{\Sdpv}{\ensuremath{S_\text{dpv}} }
\newcommand{\fracpar}[2]{\frac{\partial #1}{\partial #2}}
\newcommand{\pref}{p_\text{ref}}
\newcommand{\surf}{\text{surf}}
\newcommand{\nosurf}{\text{nosurf}}

\setlength{\parindent}{0cm}
\setlength{\parskip}{5mm}

\begin{document}

\title[surfactant model]{The surfactant model in MRST}

\maketitle

\begin{abstract}
  In this document, we describe briefly the surfactant model which is implemented in MRST. 
\end{abstract}


Our starting point is the mass conservation equations for oil, water, and surfactant, which are given by
\begin{equation}
  \label{eq:conservation_laws1}
  \frac{\partial}{\partial t} (\rho_{\alpha}\phi S_\alpha) +
  \nabla\cdot(\rho_{\alpha}\vec{v}_\alpha)  = 0, \qquad \alpha\in\{w,o\}, 
\end{equation}
and
\begin{equation}
  \label{eq:conservation_laws2_withads}
  \frac{\partial}{\partial t} (\rho_{w}\phi S_w c)+
  \fracpar{}{t}\left(\rho_{r, \text{ref}}(1-\phi_\text{ref})\cads\right)+\nabla\cdot(c
  \rho_{w} \vec{v}_{wp})  = 0.
\end{equation}

Let us describe the terms involved in \eqref{eq:conservation_laws1} and
\eqref{eq:conservation_laws2_withads}. The values $\rho_{\alpha}$, $\vec{v}_\alpha$, and $S_\alpha$
denote density, velocity, and saturation of the phase $\alpha$. The porosity is denoted by $\phi$
and is assumed to be a function $\phi(p)$ of pressure only, $c$ is the surfactant concentration, and
$\vec{v}_{wp}$ the velocity of water containing diluted polymer. The quantity $c^a$ denotes the mass
of surfactant which is adsorbed to a unit mass of rock. Sources and sinks may be included in a
manner equivalent to boundary conditions, and are left out of the above equations.

Let us describe how these quantities are determined. The fluxes $\vec{v}_\alpha$ are given by
\begin{equation*}
  \vec{v}_\alpha = \frac{k_{\alpha}(s_\alpha, c)}{\mu_{\alpha}(p, c)}\grad p_\alpha,
\end{equation*}
where the relative permeabilities $k_{\alpha}$ depends on the saturations and the surfactant
concentrations and the viscosities depend on the surfactant concentration. The phase pressures are
given by the relation
\begin{equation}
  p_o = p_w + p_c(s_w, c) 
\end{equation}
where the capillary pressure function $p_c$ depends on the water saturation and
surfactant concentration. The model for relative permeability and viscosity
follow Eclipse's implementation, see \cite{eclipse, jorgensen}. First, we
introduce the interfacial tension $\sigma$. At the pore level, the effect of
surfactant is to modify the interfacial tension so that $\sigma$ is a function
of the surfactant concentration, $\sigma(c)$. At the upscaled level, the change in
relative permeability depends on the capillary number $N_c$ which measures
the ratio between the viscous and capillary forces and is defined as
\begin{equation}
  \label{eq:defNc}
  N_c = \frac{\abs{K\grad p}}{\sigma}.
\end{equation}
Thus, the relative permeability is given as $k_{\alpha}(s_\alpha, N_c)$. The values of
$k_{\alpha}(s_\alpha, N_c)$ are computed by interpolating the relative permeability curves for
minimal and maximal $N_c$, in the following way. Let $N_c^{\nosurf}$ and $N_c^{\surf}$denote the
minimal and maximal values of the capillary numbers for which the relative permeability curves are
given as
\begin{equation*}
  k_{\alpha}^{\nosurf}(s_\alpha)\quad\text{and}\quad k_{\alpha}^{\surf}(s_\alpha).
\end{equation*}
We denote the residual saturation values, for each phase, as $s_{\alpha. res}^{\nosurf}$ and
$s_{\alpha. res}^{\surf}$. We use a logarithmic interpolation and we define the interpolation factor
$m$ as
\begin{equation} 
  \label{eq:defm}
  m(N_c) = \frac{\log_{10}(N_c) - \log_{10}(N_c^{\nosurf})}{\log_{10}(N_c^{\surf}) - \log_{10}(N_c^{\nosurf})},
\end{equation}
which belongs to $[0,1]$. The residual saturation values for the relative permeabilities
$k_{\alpha}(s_\alpha, N_c)$ is obtained by interpolation
\begin{equation}
  \label{eq:srhat}
  s_{\alpha,res}^{N_c} = m(N_c)\,s_{\alpha,res}^{\surf} + (1 - m(N_c))\,s_{\alpha,res}^{\nosurf}.
\end{equation}
The relative permeabilities $k_{r,\alpha}^{\surf}$ and $k_{r,\alpha}^{\nosurf}$
are rescaled to match these endpoints. We detail the construction only for
$k_{r,\alpha}^{\surf}$ as the construction for $k_{r,\alpha}^{\nosurf}$ is
completely similar. We define the function $k_{r,\alpha}^{\surf}$ as
\begin{equation}
\label{eq:rescaledkr}
k_{r,\alpha}^{\surf}(s_\alpha, N_c) = k_\alpha^{\surf}(s_\alpha^{\surf}(N_c)),
\end{equation}
where $k_\alpha^{\surf}$ is a given function (see above) and
$s_\alpha^{\surf}(N_c)$ is defined through the relation
\begin{equation}
  \label{eq:defsnc}
  \frac{s_\alpha^{\surf}(N_c) - s_{\alpha,res}^{\surf}}{1 - s_{\alpha,res}^{\surf} - s_{\alpha',res}^{\surf}} =   \frac{s_\alpha - s_{\alpha,res}^{N_c}}{1 - s_{\alpha,res}^{N_c} - s_{\alpha',res}^{N_c}}.
\end{equation}
In \eqref{eq:defsnc}, $\alpha'$ denotes the phase which is not the phase
$\alpha$. Similarly we define $k_{\alpha}^{\nosurf}$. Finally, we define the
relative permeability $k_{\alpha}$ by interpolation using the coefficient
$m(N_c)$, that is,
\begin{equation}
  \label{eq:defkralpha}
  k_{\alpha}(s, N_c) = m(N_c)\,k_{\alpha}^{\surf}(s_\alpha, N_c) + (1 - m(N_c))\,k_{\alpha}^{\nosurf}(s_\alpha, N_c).
\end{equation}

The change in viscosity is only modeled for the water phase by introducing a
multiplier, which is a function of the surfactant concentration. We have
\begin{equation}
  \label{eq:defmuw}
  \mu_w(p, c) = \frac{\mu_w^0(p)}{\mu_w^0(\pref)}\mu_{s}(c),
\end{equation}
where $\mu_w^0$ denotes the viscosity without surfactant and $\mu_s(c)$ denotes
the viscosity as a function of surfactant concentration for the pressure
$\pref$. For the adsorption term, we use an approach based on Langmuir isotherm
so that the adsorbed concentration is a function of the surfactant
concentration, that is of the from $\cads(s)$. We consider the reversible case
and the case without desorption where the term $\cads$ is replaced by
$\hat\cads$ which denotes the maximum value that the adsorption concentration
has reached. More precisely, if we define
\begin{equation}
  \label{eq:chatadsmax}
  \chatads_{\max}(t,x) = \max_{t'<t}(\chatads(t',x)),
\end{equation}
then we have
defined as
\begin{equation}
  \label{eq:defchatads}
  \chatads(t, x) = \max(\cads(c(t,x)), \chatads_{\max}(t,x)).
\end{equation}

\begin{table}[h]
  \centering
  \begin{tabular}[t]{|l|l|}
    \hline
    Keyword&Description\\
    \hline
    SURFACT& Model initialization\\
    \hline
    SURFST& IFT data\\
    \hline
    SURFVISC& Viscosity modifier\\
    \hline
    SURFCAPD& Capillary de-saturation\\
    \hline
    SURFADS& Surfactant adsorption\\
    \hline
    SURFROCK& Rock properties\\
    \hline
    SURFNUM& Grid region specification\\
    \hline
    WSURFACT& Injected concentration\\
    \hline
  \end{tabular}
  \caption{List of Eclipse keyword that have been implemented.}
  \label{tab:eclipsekeyword}
\end{table}

\printbibliography

\end{document}

%  LocalWords:  surfactant
