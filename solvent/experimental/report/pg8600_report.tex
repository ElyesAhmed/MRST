\documentclass[11pt, a4paper]{article}
\usepackage[utf8]{inputenc} % Required for inputting international
                            % characters
\usepackage[T1]{fontenc}    % Output font encoding for international
                            % characters

\usepackage[osf,sc]{mathpazo}
\usepackage[scaled=0.90]{helvet}
\usepackage[scaled=0.85]{beramono}

\usepackage{natbib}
\bibliographystyle{unsrtnat}
\usepackage{geometry}
\usepackage{graphicx}
\usepackage{tikz}
\usetikzlibrary{calc}
\tikzset{every picture/.style={remember picture}}

\usepackage{xcolor} % NTNU blue
\definecolor{ntnuBlue}{cmyk}{1.0, 0.75, 0.0, 0.05}

\usepackage{titlesec} % Coloring titles

\titleformat{\section}%
{\color{ntnuBlue}\normalfont\Large\bf}%
{\color{ntnuBlue}\thesection}%
{1em}%
{}%

\titleformat{\subsection}%
{\color{ntnuBlue}\normalfont\bf\itshape}%
{}%
{}%
{}%


\usepackage{etoolbox}
\usepackage[font=scriptsize]{subfig}

\makeatletter
\patchcmd{\ttlh@hang}{\parindent\z@}{\parindent\z@\leavevmode}{}{}
\patchcmd{\ttlh@hang}{\noindent}{}{}{}
\makeatother

% \usepackage{soul}
% \newcommand{\com}[1]{\textcolor{red}{#1}}
% \setul{0}{1mm}
% \setstcolor{red}

\usepackage{hyperref} % Required for customising links and the PDF
\hypersetup{pdfpagemode={UseOutlines}, bookmarksopen=true,
  bookmarksopenlevel=0, hypertexnames=false,
  colorlinks=true, % Set to false to disable coloring links
  citecolor=ntnuBlue, % The color of citations
  linkcolor=ntnuBlue, % The color of references to document elements
                      % (sections, figures, etc)
  urlcolor=ntnuBlue, % The color of hyperlinks (URLs)
  pdfstartview={FitV}, unicode, breaklinks=true, }

\usepackage{amsmath}
\usepackage{amsfonts, amssymb, amsthm}

\DeclareMathOperator\real{{\mathbb{R}}}
\DeclareMathOperator{\dd}{d}
\DeclareMathOperator{\supp}{supp}
\DeclareMathOperator{\eff}{eff}
\renewcommand{\vec}[1]{\boldsymbol{#1}}

\usepackage{algorithm}
\usepackage[noend]{algpseudocode}

\usepackage{enumitem}


\title{\bfseries \textcolor{ntnuBlue}{PG8600 \\ \vspace{5pt} \Huge{Implementation of a Solvent Model Using MRST}}}
\author{Øystein Strengehagen Klemetsdal
  \thanks{oystein.klemetsdal@ntnu.no}}

\begin{document}

\maketitle

\begin{abstract}

\end{abstract}

\section{Introduction}

Miscible flooding is a proven, economically viable process that significantly
increases the oil recovery factor for a wide range of reservoir types. The main
purpose of miscible gas injection is to recover as as much as possible of the
trapped residual oil after a waterflood. In addition, solvent may also displace
oil in the upper regions of the reservoir poorly swept by the water due to
gravity.

There are three main solvent-injection strategies:
\begin{enumerate}[label=\emph{(\roman*)}]
\item Slug injection: Usually 0.2-0.4 hydrocarbon pore-volumes (HCPV) of solvent
  that, in turn, is displaced by water or dry solvent.
\item Water-alternating-gas (WAG): Alternating injection of small volumes
  (0.01-0.04 HCPV) of water and solvent. The total amount usually ranges from
  0.2 to 0.6 HCPV. The final drive fluid is usually water. The small water slugs
  decreases solvent mobility and leads to increased solvent sweep efficiency
\item Gravity-stable injection: For some pinnacle reefs and stepply dipping
  reservoirs with high vertical communication, it is advantageous to inject
  less-dense solvent at the top of the reservoir in a gravity-stable
  displacement process.
\end{enumerate}

\paragraph{Solvent mechanisms}

There are several mechanisms involved in solvent displacement process. The most
obvious is the direct miscible displacement of oil by solvent along
high-permeability paths. Additionally, part of the oil initially bypassed by
solvent may later be recovered by oil swelling, which occurs as the solvent
dissolves the oil, or by extraction of oil into solvent. This takes place as the
solvent continues to flow past the previously bypassed oil. On a field scale,
the single most important factor affecting the performance of a miscible flood
is usually the sweep efficiency.

\paragraph{Determining miscibility}

In a true miscible displacement, injected and displaced phases mix in all
proportions, without forming interfaces. This implies that injected solvent
would eventually displace all residual oil form the pores it invades. However,
most solvents does not fulfill this single-phase definition, and form two
distinct phases over a broad range of mixtures and pressures when combined with
reservoir oil. 



An injected fluid sets up a miscible displacement if there is no phase boundary
or interface between the injected fluid and the reservoir oil. The advantage of
this is a much higher recovery factor than for a conventional immiscible
displacement such water flooding; an area swept by a miscible fluid typically
leaves very little residual oil.

In a miscible gas injection process, we typically have a high adverse mobility
ratio. This results in an unstable flow, which eventually leads to viscous
fingers. In addition, gravitational fingering may occur if the density
differences are large. This means that the total displacement of the oil in the
swept regions may not lead to a high overall efficiency. On the other hand,
mixing of miscible fluids exerts a considerable damping effect on the growth of
viscous and gravity fingers. Thus, a good solvent simulation model must thus
account both these effects.

The first simulation attempts of incompressible miscible displacement was based
on a direct solution of the convection-diffusion equations for the local
concentration of each miscible component. However, unless the grid is
sufficiently fine, such models fails to capture the fine-scale structure of the
fingers, resulting in overly optimistic recovery forecasts. Moreover, the
numerical diffusion of finite-difference models will mask the true mixing of
miscible components in a coarse grid. Altogether, this imposes an unacceptably
high grid resolution.

The ECLIPSE implementation we will attempt to match in this work uses
intermediate configurations of a four-component system (water, oil, reservoir

gas and injected solvent), or a three-component system (water, oil and solvent
gas) or a two-component system (oil and solvent). The following miscibility
assumptions are made:

\begin{enumerate}[label=\emph{(\roman*)}]
\item In the regions containing only solvent and reservoir oil (possibly containing dissolved gas),
  the solvent and reservoir oil components are assumed to be miscible in all proportions, so that
  only one hydrocarbon phase exists in the reservoir. We use the same relative permeabilites as for
  a two-phase water/hydrocarbon system.
\item In regions containing only reservoir oil and gas, the gas and oil components be immiscible,
  and behave in a traditional black-oil manner. In regions containing both dry gas and solvent, an
  intermediate behavior is assumed to occur.
\end{enumerate}

\subsection*{The Todd-Langstaff model}

The mixing of two miscible fluids can be described by the Todd-Langstaff model
\cite{todd1972development}. This is an empirical treatment of the effects of physical dispersion
between the miscible components in the hydrocarbon phase. The model uses an empirical parameter
$\omega$, with a value between $0$ and $1$ to represent the size of the dispersed zone between the
gas and oil components in each grid cell. This effectively controls the degree of mixing: A value of
$1$ corresponds to the case where the size of the dispersed zone is much grater than a typical grid
cell, while a value of $0$ corresponds to the effect of a negligibly thin dispersed zone. This means
that for $\omega = 1$, the hydrocarbon components can be considered to be fully mixed in each cell,
while for $\omega = 0$, the hydrocarbons will have the density and values of the pure components.

\section{Model formulation}

\subsection*{Relative permeabilities}

In regions of small solvent saturation, the displacement is immiscible. In the usual black-oil
model, the relative permeabilities for water, oil and gas phases are
\begin{equation*}
  k_{rw} = k_{rw}(S_w), \quad k_{rg} = k_{rg}(S_g), \quad k_{ro} = k_{ro}(S_w, S_g).
\end{equation*}
When two gas phases are present, we assume that the total relative permeability of the gas phase is
a function of the total gas saturation,
\begin{equation*}
  k_{rgt} = k_{rg}(S_g + S_s),
\end{equation*}
where $S_s$ is the solvent saturation. Then, the relative permeability of each gas component is
taken as a function of the fraction of each gas component within the gas phase:
\begin{equation}
  \label{eq:relperm_gas_solvent}
  k_{rs} = k_{rgt}k_{rfs}(F_s), \quad k_{rg} = k_{rgt}k_{rfg}(F_g), \quad \text{where}
  \quad F_\alpha = \frac{S_\alpha}{S_g + S_s}.
\end{equation}
Typically, $k_{rfs}$ and $k_{rfg}$ are linear functions, with $k_{rf\alpha}(0) = 0$ and $k_{rf\alpha}(1) = 1$.

In regions where solvent is displacing oil, the hydrocarbon displacement is
miscible. However, the two-phase character of the water/hydrocarbon displacement
needs to be taken into account. The relative permeabilities are the
\begin{equation}
  \label{eq:miscible_relperms}
  k_{ro} = \frac{S_o}{S_n}k_{rn}(S_n), \quad k_{rgt} = \frac{S_s + S_g}{S_n}k_{rn}(S_n),
\end{equation}
where $S_n = S_g + S_s + S_o$ is the total hydrocarbon saturation, $k_{rgt}$ is the total relative
permeability of the gas and solvent, and $k_{rn}(S_n)$ is the relative permeability of hydrocarbon
to water. The gas and solvent relative permeabilites are then found from
\eqref{eq:relperm_gas_solvent} using this $k_{rgt}$. It is also possible to modify the linear
miscible relperms using multipliers:
\begin{equation*}
  k_{ro} = M_{kro}\left(\frac{S_o}{S_n}\right)k_{rn}(S_n), \quad k_{rgt} = M_{krsg}\left(\frac{S_s + S_g}{S_n}\right)k_{rn}(S_n),
\end{equation*}

\subsection*{Transition between miscible and immiscible phases}

The transition between the miscible and immiscible case is handled by a
miscibility function $M$, which is a function of the solvent fraction in the gas
phase. The function is tabulated between $0$ and $1$, where $M = 0$ and $M = 1$
implies immiscible and miscible displacement, respectively. The transition
algorithm can be described as follows:
\begin{enumerate}[label=\emph{(\roman*)}]
\item Scale the relperm endpoints using $M$:
  \begin{equation*}
    S_{or} = S_{orm}M + S_{ori}(1-M), \quad S_{sgr} = S_{sgrm}M + S_{sgri}(1-M),
  \end{equation*}
  where $S_{orm}$ and $S_{ori}$ are the miscible and immiscible residual oil saturations,
  respectively, and $S_{sgrm}$ and $S_{sgri}$ are the miscible and immiscible residual solvent + gas
  saturations.
\item Calculate the miscible and immiscible relperms at the scaled saturations using the new end
  points. The relperm is again an interpolation:
  \begin{equation*}
    k_r = k_{rm}M + k_{ri}(1-M),
  \end{equation*}
  where $k_{rm}$ and $k_{ri}$ are the scaled miscible and immiscible relperms.
\end{enumerate}
The equations \eqref{eq:miscible_relperms} are then modified to take into account these new
saturation end points:
\begin{equation*}
  k_{ro} = \frac{S_o - S_{or}}{S_n - S_{gc} - S_{or}}k_{rn}(S_n), \quad k_{rgt} = \frac{S_s + S_g - S_{gc}}{S_n - S_{gc} - S_{or}}k_{rn}(S_n),
\end{equation*}
where $S_{gc}$ is the critical oil-to-gas saturation.

\subsection*{Effect of pressure on relative permeabilities and capillary pressure misciblity}

In many miscible displacements, the solvent is only miscible with reservoir oil
at high pressure. The solvent/oil capillary pressure typically reduce with
increasing pressure, and the two fluids can only be considered miscible once the
capillary pressure is zero. The pressure dependence can be modeled by
introducing a pressure-dependent miscibility function $M_p$, used to interpolate
between immiscible and miscible values of the PVT, relperm and capillary
pressure data. Neglecting PVT behavior, we have that for the relperms and
saturation end points, the effect of pressure miscibility is combined with the
solvent saturation effects, so that the multiplier $M$ is now replaced by
$M_T = MM_p$. The capillary pressure is interpolated as
\begin{equation*}
  P_{cog} = M_p P_{cog,m} + (1-M_p)P_{cog,i}, \quad \text{where} \quad P_{cog,m} = P_{cog}(S_g), \quad P_{cog,i} = P_{cog}(S_g + S_s).
\end{equation*}

\subsection*{Effect of water saturation}

A feature of miscible gas injection that may be modeled is the screening effect
of high water saturation, which reduces the contact between the miscible gas and
the in-place oil. The effective residual oil saturation to a miscible gas drive
is found to increase with increasing water saturation. This process is modeled
by introducing an effective residual oil saturation $S_{or}(S_w)$, which depends
on the water saturation. A mobile oil saturation is then found by
\begin{equation*}
  S_o^* = \max\{S_o - S_{or}, 0\}.
\end{equation*}
For completeness, we also introduce a critical gas saturation $S_{gc}(S_w)$, and define the mobile
gas saturation
\begin{equation*}
  S_g^* = \max\{S_g - S_{gc}, 0\}.
\end{equation*}
The mobile oil and gas saturations are used to determine the miscible component relperms and
effective viscosities and densities in each grid cell.

\subsection*{Viscosity model}

The full three hydrocarbon components mixing calculation can be thought of as
two separate miscible displacements: Gas/solvent and solvent/oil. The effective
viscosities of the hydrocarbon components follow by the Todd-Longstaff model:
\begin{equation*}
  \mu_{o, \eff} = \mu_o^{1-\omega}\mu_{mos}^\omega, \quad
  \mu_{s, \eff} = \mu_s^{1-\omega}\mu_{m}^\omega, \quad
  \mu_{g, \eff} = \mu_g^{1-\omega}\mu_{msg}^\omega,
\end{equation*}
where $\mu_\alpha$ is the component viscosity of component $\alpha$, and $\omega$ is the mixing
parameter. Further, $\mu_{mos}$ is the fully mixed viscosity of oil + solvent, $\mu_{msg}$ is the
fully mixed viscosity of solvent + gas, $\mu_m$ is the fully-mixed viscosity of oil + solvent + gas,
and are defined using the $1/4$th-power fluid mixing rule:
\begin{align*}
  \mu_{mos} & = \frac{\mu_o \mu_s}{\left(\frac{S_o'}{S_{os}'}\mu_s^{1/4} + \frac{S_s'}{S_{os}'}\mu_o^{1/4}\right)^4} \\
  \mu_{msg} & = \frac{\mu_s \mu_g}{\left(\frac{S_s'}{S_{sg}'}\mu_g^{1/4} + \frac{S_g'}{S_{sg}'}\mu_s^{1/4}\right)^4} \\
  \mu_{m} & = \frac{\mu_o \mu_s \mu_g}{\left(\frac{S_o'}{S_{n}'}\mu_s^{1/4}\mu_g^{1/4}
            + \frac{S_s'}{S_{n}'}\mu_o^{1/4}\mu_g^{1/4}
            + \frac{S_g'}{S_{n}'}\mu_o^{1/4}\mu_s^{1/4}\right)^4},
\end{align*}
where the mobile saturations are defined by
\begin{align*}
  S_o' & = S_o - S_{or}, \quad S_g' = S_o - S_{gc}, \quad S_s' = S_s - S_{gc}, \\
  S_n' & = S_o' + S_g' + S_s', \quad S_{os}' = S_o' + S_s', \quad S_{sg}' = S_s' + S_g'.
\end{align*}

\subsection*{Density model}

The effective component viscosities are used to find effective saturation fractions:
\begin{subequations}
  \begin{align}
    \label{eq:sat_frac_oe}
    \left[\frac{S_o}{S_n}\right]_{oe} &= \frac{\mu_o^{1/4}\left(\mu_{o,\eff}^{1/4} - \mu_s^{1/4}\right)}{\mu_{o,\eff}^{1/4}\left(\mu_o^{1/4} - \mu_s^{1/4}\right)} \\
    \label{eq:sat_frac_ge}
  \left[\frac{S_o}{S_n}\right]_{ge} &= \frac{\mu_s^{1/4}\left(\mu_{g,\eff}^{1/4} - \mu_g^{1/4}\right)}{\mu_{g,\eff}^{1/4}\left(\mu_s^{1/4} - \mu_g^{1/4}\right)} \\
  \left[\frac{S_s}{S_n}\right]_{se} &= \frac{\mu_s^{1/4}\left(S_{gf}\mu_o^{1/4} - S_{of}\mu_g^{1/4}\right)
                                      - \mu_o^{1/4}\mu_g^{1/4}\frac{\mu_s^{1/4}}{\mu_{s,\eff}^{1/4}}}{\mu_s^{1/4}\left(S_{gf}\mu_o^{1/4} - S_{of}\mu_g^{1/4}\right) - \mu_o^{1/4}\mu_g^{1/4}},
\end{align}
\end{subequations}
where
\begin{equation*}
  S_{of} = S_o'/S_{og}, \quad S_{gf} = S_g'/S_{og}', \quad S_{og} = S_o'/S_g'.
\end{equation*}
The effective densities are then found by
\begin{align*}
  \rho_{o,\eff} & = \rho_o\left[\frac{S_o}{S_n}\right]_{oe} + \rho_s\left(1 - \left[\frac{S_o}{S_n}\right]_{oe}\right), \\
  \rho_{g,\eff} & = \rho_s\left[\frac{S_o}{S_n}\right]_{ge} + \rho_g\left(1 - \left[\frac{S_o}{S_n}\right]_{ge}\right), \\
  \rho_{s,\eff} & = \rho_s\left[\frac{S_s}{S_n}\right]_{se} + (\rho_gS_{gf} + \rho_oS_{of})\left(1 - \left[\frac{S_s}{S_n}\right]_{se}\right),
\end{align*}
Note that \eqref{eq:sat_frac_oe} is singular for unit mobility ratio $\mu_o = \mu_s$, and that
\eqref{eq:sat_frac_ge} is singular for $\mu_s = \mu_g$. In this case, the effective densities are
computed as follows: First, we define the mixture density
\begin{equation*}
  \rho_m = \rho_o \frac{S_o}{S_n} + \rho_g \frac{S_g}{S_n} + \rho_s \frac{S_s}{S_n},
\end{equation*}
and use this to define the effective densities
\begin{equation*}
  \rho_{\alpha,\eff} = (1-\omega) \rho_\alpha + \omega \rho_m.
\end{equation*}

\subsection*{Pressure effect on viscosity and density miscibility}

The pressure dependence of the transition between miscible and immiscible flow can be modeled for
PVT data by
\begin{equation*}
  b_\alpha = b_{\alpha,m} M_p + b_{\alpha,i}(1-M_p) \quad  \text{ and } \quad
  \frac{b_\alpha}{\mu_\alpha} = \frac{b_{\alpha,m}}{\mu_{\alpha,m}} M_p + \frac{b_{\alpha,i}}{\mu_{\alpha,i}}(1-M_p),
\end{equation*}
where $b_{\alpha,m}$ and $b_{\alpha,i}$ are the miscible and immiscible formation volume factors,
respectively. 


\section{Mass balance}

Alternative notation:

\begin{align*}
  \left[\frac{S_o}{S_n}\right]_{oe} & = \frac{(\mu_o/\mu_s)^{1/4} - (\mu_o/\mu_{o,\eff})^{1/4}}{(\mu_o/\mu_s)^{1/4} - 1} = \xi_o \\
  \left[\frac{S_o}{S_n}\right]_{ge} & = \frac{(\mu_s/\mu_g)^{1/4} - (\mu_s/\mu_{g,\eff})^{1/4}}{(\mu_s/\mu_g)^{1/4} - 1} = 1-\xi_g, \\
  \left[\frac{S_s}{S_n}\right]_{se} &= \frac{(\mu_s/\mu_g)^{1/4}\frac{S_g}{S_o + S_g} + (\mu_s/\mu_o)^{1/4}\frac{S_o}{S_o + S_g} - (\mu_s/\mu_{s,\eff})^{1/4}}
{(\mu_s/\mu_g)^{1/4}\frac{S_g}{S_o + S_g} + (\mu_s/\mu_o)^{1/4}\frac{S_o}{S_o + S_g} - 1} = \xi_s.
\end{align*}
In this notation, mass balance requires
\begin{align*}
  \rho_w S_w +& \rho_o S_o + \rho_g S_g + \rho_s S_s = \rho_{w, \eff} S_w + \rho_{o, \eff} S_o + \rho_{g, \eff} S_g + \rho_{s, \eff} S_s \\
             = & S_w \rho_w + S_o \left(\rho_o \xi_o + (1-\xi_o)\rho_s\right)
               + S_g \left(\rho_g \xi_g + (1-\xi_g)\rho_s\right) \\
             & + S_s \left(\rho_s \xi_s + (1-\xi_s)\left[\rho_o \frac{S_o}{S_o + S_g} + \rho_g \frac{S_g}{S_o + S_g}\right]\right),
\end{align*}
which may be rewritten
\begin{align*}
  S_o(1-\xi_o)(\rho_o - \rho_s) + S_g(1-\xi_g)(\rho_g - \rho_s) +
  S_s(1-\xi_s)\left(\rho_s - \rho_o \frac{S_o}{S_g + S_o} - \rho_g \frac{S_g}{S_g + S_o}\right) = 0
\end{align*}
 

\begin{align*}
  \mu_{o,\eff}
  & = \frac{\mu_o^{1-\omega}\mu_o^\omega\mu_s^\omega}{\left(\frac{S_o}{S_o + S_s}\mu_s^{1/4} + \frac{S_s}{S_o + S_s}\mu_o^{1/4}\right)^{4\omega}} \\
  & = \frac{\mu_o\mu_s^\omega}{\left(\frac{S_o}{S_o + S_s}\mu_s^{1/4} + \frac{S_s}{S_o + S_s}\mu_o^{1/4}\right)^{4\omega}} \\
  & = \frac{\mu_o}{\left(\frac{S_o}{S_o + S_s} + \frac{S_s}{S_o + S_s}\left(\frac{\mu_o}{\mu_s}\right)^{1/4}\right)^{4\omega}} \\
\end{align*}

\begin{align*}
  \left(\frac{\mu_o}{\mu_{o,\eff}}\right)^{1/4} = \left(\frac{S_o}{S_o + S_s} + \frac{S_s}{S_o + S_s}\left(\frac{\mu_o}{\mu_s}\right)^{1/4}\right)^{\omega}
\end{align*}

Likewise:

\begin{align*}
  \mu_{s,\eff}
  & = \frac{\mu_s}{\left(\frac{S_o}{S_n} \left(\frac{\mu_s}{\mu_o}\right)^{1/4}+ \frac{S_s}{S_n} + \frac{S_g}{S_n}\left(\frac{\mu_s}{\mu_g}\right)^{1/4}\right)^{4\omega}} \\
\end{align*}
and
\begin{align*}
  \left(\frac{\mu_s}{\mu_{s,\eff}}\right)^{1/4} = \left(\frac{S_o}{S_n} \left(\frac{\mu_s}{\mu_o}\right)^{1/4}+ \frac{S_s}{S_n} + \frac{S_g}{S_n}\left(\frac{\mu_s}{\mu_g}\right)^{1/4}\right)^{\omega}
\end{align*}




\begin{align*}
  S_o(1-\xi_o)(\rho_o - \rho_s) + S_g\xi_g(\rho_g - \rho_s) + S_s(1-\xi_s)\left(\rho_s - \rho_o \frac{S_o}{S_g + S_o} - \rho_g \frac{S_g}{S_g + S_o}\right) = 0
\end{align*}

\begin{small}
  \bibliography{refs}
\end{small}

\end{document}


%%% Local Variables:
%%% mode: latex
%%% TeX-master: t
%%% End:
