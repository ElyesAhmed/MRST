\documentclass[11pt,fleqn]{amsart}
\usepackage{graphicx,amsmath,amssymb,a4wide}
\usepackage[numbers,sort&compress]{natbib}
\usepackage[hang]{subfigure}

\renewcommand{\div}{\ensuremath{\operatorname{div}}}
\newcommand{\vect}[1]{\boldsymbol{#1}}
\newcommand{\mat}[1]{\boldsymbol{#1}}
\newcommand{\tens}[1]{\boldsymbol{\mathsf{#1}}}
\newcommand{\transp}[1]{{#1}^{\ensuremath{\mathsf{T}}}}

\newtheorem{remark}{Remark}

\begin{document}

\title[Local flux mimetic in Matlab]{Implementation of local flux mimetic in Matlab}%
\author[J.~R.~Natvig]{Jostein R. Natvig}%
\address{SINTEF ICT, Applied Mathematics, P.O.~Box 124 Blindern, N--0314 Oslo, Norway}
\email[]{\{Jostein.R.Natvig\}@sintef.no}


%\begin{abstract}
%\end{abstract}

\maketitle

\section{Introduction}
The implementation of the local flux mimetic method in MRST
\citep{MRST} is straightforward.  The most computationally demanding
step is the inversion of the mixed mass matrix which has much higer
diimension than the regular mixed or hybrid mass matrix.  From an
implementation poit of view, the construction of a numbering of the
sub-faces with corresponding mappings to nodes, faces and cells is the
most complex.  Furthermore, the choosing to use this scheme as either
a mimetic method or a multi-point flux scheme puts certain limitations
on its use. In the former case, the scheme is quite coumputationally
demanding to assemble, store and handle, while in the latter case, the
inclusion of gravity and forcings (bc, wells, src) is not solved
(yet).

\section{Reducing mixed hybrid system to cell pressure system}
Consider the elliptic equation for pressure, 
\begin{align}
  \label{eq:pressure}
  \vect{v} + \mat{K}\nabla p  &=  0,\\
  \nabla\cdot\vect{v}        &= q,
\end{align}
where $\mat{K}$ is the permeability tensor, $p$ is the fluid pressure
and $\vect{v}$ is the fluid velocity.  To discretise
\eqref{eq:pressure} using a local-flux mimetic method, we start by
decribing the mass matrix $\mat{B}$.  If $\mat{B}$ is exact for linear
pressures, we can expect to get a consistent discretisation of
\eqref{eq:pressure}.  Thus, for any linear pressure $p = p_o +
\vect{c}\cdot\vect{x}$, we get the relation
\begin{equation}
  \label{ep:exactlinear}
  \mat{N}\vect{c} = -\mat{K}\transp{\mat{R}}\vect{c},
\end{equation}
where $\mat{N}$ is the $n\times d$ matrix of face normals and
$\mat{R}$ is the corresponding matrix of vectors from the cell centre,
where the cell pressure is computed, to the face centres, where the
face pressures are computed.  To get a local expression for the face
flux, we make the following choices: We split each $k$-gonal face into
$k$ sub-faces, each with one corner belonging to the face.  For each
sub-face of face $i$ , we use $\mat{N}_i$ and $\mat{R}_i$ as normal
and distance. Then, we require that $\mat{B}$ be block-diagonal with
one block for each corner in the cell.  For a regular grid, each such
corner will be the intersection of $d$ sub-faces, and the
corresponding block in \eqref{eq:exactlinear} will be unique.  Note
that this matrix is not symmetric, and $\langle\vect{F},\vect{G}\rangle =
\transp{\vect{F}}\mat{B}\vect{G}$ is not an inner-product but a
bilinear form on sub-face-fluxes.

To derive a cell-centered scheme from the definition of $\mat{B}$, we
start with the usual mimetic mixed hybrid formulation
\begin{equation}
  \label{eq:hybLinSys}
   \begin{bmatrix}
     \mat{B} & \mat{C} & \mat{D}\\
     \transp{\mat{C}}  & \mat{0} & \mat{0}\\
     \transp{\mat{D}} &  \mat{0} & \mat{0}
   \end{bmatrix}
   \begin{bmatrix}
     \vect{v_H}\\ -\vect{p}\\ \vect{\pi}
   \end{bmatrix}
   =
   \begin{bmatrix}
     \vect{0} \\ \vect{q}\\ \vect{0}
   \end{bmatrix}
\end{equation}
where $\mat{C}$ is block-diagonal with blocks $\vect{e} = (1,1,\ldots,
1)$, that adds up sub-faces, and $\mat{D}$ maps from cell-wise
ordering to global ordering of sub-faces. We introduce $\mat{D_o}$
that maps hybrid velocities to mixed velocities
\begin{equation}
\label {eq:hybrid2mixed}
\vect{v_h} = \mat{D_o}\vect{v_m}.
\end{equation}
Left-multiplying the first equation in \eqref{eq:hybLinSys} by
$\transp{\mat{D_o}}$, substituting \eqref{eq:hybrid2mixed}, we get
\begin{align*}
  &\transp{\mat{D_o}}\mat{B}\mat{D_o}\vect{v_m} -
  \transp{\mat{D_o}}\mat{C}\vect{p} + 
  \transp {\mat{D_o}}\vect{\pi}=\vect{0}\\
  &\mat{C}\mat{D_o}\vect{v_m} = \vect{q}.
\end{align*}
If we can invert $\transp{\mat{D_m}}\mat{B}\mat{D_m}$, the resulting
linear system for cell pressures is 
%\begin{equation*}
%  \transp{\mat{C}}\mat{D_m}[\transp{\mat{D_m}}\mat{B}\mat{D_m}]^{-1}
%         [\transp{\mat{D_m}}\mat{C}\vect{p}-\transp{\mat{D_m}}\mat{D}\vect{\pi}] = q,
%\end{equation*}
%that reduce to 
\begin{equation}
  \transp{\mat{C}}\mat{D_o}[\transp{\mat{D_o}}\mat{B}\mat{D_o}]^{-1}
         [\transp{\mat{D_o}}\mat{C}\vect{p}-\vect{\pi_b}] = q
\end{equation}
where the boundary pressures $\vect{\pi_b}$ are the only nonzero
entries of $\transp{\mat{D_o}}\mat{D}\vect{\pi}$.  To get an
expression for the flux interms of cell-pressures, we introduce
$\mat{C} = \mat{C_2}\mat{C_1}$, where $\mat{C_1}$ is block-diagonal
with as many rows as faces in the grid, and as many columns as
sub-faces in the grid.  $\mat{C}_1\vect{v_H}$ adds the sub-face-fluxes
of each face.  The other matrix $\mat{C}_2$ is the standard mimetic
$\mat{C}$-matrix.  We can then write 
\begin{equation*}
  \vect{v} = \transp{\mat{C}_1}\mat{D_o}[\transp{\mat{D_o}}\mat{B}\mat{D_o}]^{-1}
             [\transp{\mat{D_o}}\mat{C}\vect{p}-\vect{\pi_b}] 
\end{equation*}
and 
\begin{equation*}
  \mat{T} = \transp{\mat{C}_1}\mat{D_o}[\transp{\mat{D_o}}\mat{B}\mat{D_o}]^{-1}
             \begin{bmatrix}\transp{\mat{D_o}}\mat{C}\\-\mat{I}_b\end{bmatrix}
\end{equation*}

\section{Local flux mimetic}
The crucial point in the previous section is the inversion of the
mixed mass matrix $\transp{\mat{D_m}}\mat{B}\mat{D_m}$.  The
construction of the local flux mimetic is based on two ideas that
allow a cheap inversion of this matrix.  First, the number of degrees
of freedom in the face pressures $\vect{\pi}$ and half-face velocities
$\vect{v_h}$ is increased to one pressure and velocity per face per
node. Thus, for a face with $n$ corners, there are $n$ facet pressures
and velocities , one for each facet containing exactly one face
corner.

Second, the mass matrix $\mat{B}$ defining the innerproduct between
discrete facet fluxes chosen to be block-diagonal with one $d$ x $d$
block per node in the cell.  This inner product is uniqly defined if
$d$ facets with linearly independent normals meet in each cell-node.

Inversion of the mixed mass matrix
$\transp{\mat{D_m}}\mat{B}\mat{D_m}$ amounts to inverting the diagonal
blocks corresponding to each node in the grid.  Note that this
corresponds exactly to the simplest multi-point flux approximation
called the O-method [].  However, this derivation yields a simpler
starting point for implementation on general grids.

To get a multi-point flux transmissibility $\mat{T}$, the
contributions from each facet must be added for each face in the grid.





{\bf How much does it matter if facet centriods are inaccurate?}

\bibliographystyle{abbrvnat}
\bibliography{misc}

\end{document}

%{
Copyright 2009-2018 SINTEF ICT, Applied Mathematics.

This file is part of The MATLAB Reservoir Simulation Toolbox (MRST).

MRST is free software: you can redistribute it and/or modify
it under the terms of the GNU General Public License as published by
the Free Software Foundation, either version 3 of the License, or
(at your option) any later version.

MRST is distributed in the hope that it will be useful,
but WITHOUT ANY WARRANTY; without even the implied warranty of
MERCHANTABILITY or FITNESS FOR A PARTICULAR PURPOSE.  See the
GNU General Public License for more details.

You should have received a copy of the GNU General Public License
along with MRST.  If not, see <http://www.gnu.org/licenses/>.
%}
